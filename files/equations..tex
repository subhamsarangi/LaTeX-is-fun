\documentclass{article}
\usepackage{amsmath}

\begin{document}

\section*{Key Quantum Mechanics Equations}

\subsection*{1. Schrödinger's Equation}

The time-dependent Schrödinger equation is given by:
\[
i \hbar \frac{\partial \psi(\mathbf{r}, t)}{\partial t} = \hat{H} \psi(\mathbf{r}, t)
\]
where:
\begin{itemize}
    \item \( i \) is the imaginary unit,
    \item \( \hbar \) is the reduced Planck's constant,
    \item \( \psi(\mathbf{r}, t) \) is the wave function,
    \item \( \hat{H} \) is the Hamiltonian operator.
\end{itemize}

The time-independent Schrödinger equation is:
\[
\hat{H} \psi(\mathbf{r}) = E \psi(\mathbf{r})
\]
where \( E \) is the energy eigenvalue.

\subsection*{2. Heisenberg Uncertainty Principle}

The Heisenberg uncertainty principle is expressed as:
\[
\Delta x \Delta p \geq \frac{\hbar}{2}
\]
where:
\begin{itemize}
    \item \( \Delta x \) is the uncertainty in position,
    \item \( \Delta p \) is the uncertainty in momentum.
\end{itemize}

\subsection*{3. Born Rule}

The Born rule relates the wave function to probability:
\[
P(x) = |\psi(x)|^2
\]
where \( P(x) \) is the probability density of finding a particle at position \( x \).

\subsection*{4. Commutation Relation}

The fundamental commutation relation between position and momentum operators is:
\[
[\hat{x}, \hat{p}] = i \hbar
\]
where:
\begin{itemize}
    \item \( \hat{x} \) is the position operator,
    \item \( \hat{p} \) is the momentum operator.
\end{itemize}

\subsection*{5. Pauli Exclusion Principle}

The Pauli exclusion principle states that no two fermions can occupy the same quantum state:
\[
\psi(1, 2) = -\psi(2, 1)
\]

\end{document}
